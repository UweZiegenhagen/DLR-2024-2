%!TEX TS-program = Arara
% arara: pdflatex: {shell: yes}
\documentclass[12pt,ngerman]{beamer}

\usepackage{babel}
\usepackage[T1]{fontenc}

\title{Meine erste Präsentation}
\author{Uwe Ziegenhagen}
\subtitle{Warum LaTeX cool ist}
\institute{Dante e.V.}

\begin{document}

\begin{frame}

\maketitle


\end{frame}

\begin{frame}
\frametitle{Einleitung}

\begin{itemize}
\item Hallo Welt
\item ich bin 
\item eine Präsentation
\end{itemize}


\end{frame}

\begin{frame}
\frametitle{Einleitung}

\begin{enumerate}
\item Hallo Welt
\item ich bin 
\item eine Präsentation
\end{enumerate}


\end{frame}

\begin{frame}
\frametitle{Einleitung}

\begin{description}
\item [Apfel] Hallo Welt
\item [Birne]  ich bin 
\item [Pflaume] eine Präsentation
\end{description}


\end{frame}

\begin{frame}

\begin{figure}
\begin{center}
\includegraphics[width=0.75\textwidth]{Bilder/Katze}
\caption{Hallo}
\end{center}
\end{figure}

\end{frame}

\begin{frame}
\frametitle{Animationen}

\begin{itemize}
\item<2> Hallo
\item<-3> Welt
\item<1,3> Ich
\item<1-> Bin
\item ein
\item Text
\end{itemize}
\end{frame}

\begin{frame}
\frametitle{sdfadfasdfsadf}
\framesubtitle{sdfadfasdfsadf}


\begin{columns}
\begin{column}{0.49\textwidth}
\begin{itemize}
	\item 
	\item 
	\item 
	\item 
	\item 
	\item 
	\end{itemize}
\end{column}
\begin{column}{0.49\textwidth}
\begin{itemize}
	\item 
	\item 
	\item 
	\item 
	\item 
	\item 
	\end{itemize}
\end{column}
\end{columns}

\end{frame}



\end{document}

