%!TEX TS-program = Arara
% arara: pdflatex: {shell: yes}

\documentclass[12pt,a4paper]{article}
\usepackage[T1]{fontenc}
\usepackage{nccmath}
\usepackage{fourier}
%\usepackage[left=2.5cm,right=2.5cm,top=2cm,bottom=2cm,landscape]{geometry} % Updated 
\usepackage[left=2.5cm,right=2.5cm,top=2cm,bottom=2cm]{geometry} % Updated margins for improved readability
\usepackage{siunitx}
\usepackage{fancyhdr}
\usepackage{chemformula}
\usepackage{booktabs}

%\sisetup{locale=EN}
\sisetup{per-mode = symbol-or-fraction}
\sisetup{separate-uncertainty=true}
\DeclareSIUnit\year{a}
\DeclareSIUnit\clight{c}
\def\gray#1{\textcolor{gray}{#1}}
\newcolumntype{A}{ >{$} r <{$} @{} >{${}} l <{$} }
\def\linebreak{\newline \newline}
\def\largelinebreak{\newline \newline \newline}
\newcommand{\myquad}[1][1]{\hspace*{#1em}\ignorespaces}

\usepackage{titlesec}
\titleformat{\subsubsection}[block]{\large\bfseries\filcenter}{}{1em}{}

\newcommand{\deen}[2]{#1 / \textcolor{red}{#2}}

%\AtBeginDocument{\RenewCommandCopy\qty\SI} 

\begin{document}
\pagestyle{fancy}
\lhead{Formelsammlung Solartechnik \\ \gray{Formulary Solar Technology}}
\rhead{Wintersemester 24/25 \\ \gray{Winter term 24/25}}

\title{\Large{Formelsammlung Solartechnik / \gray{Formulary Solar Technology}}}
\author{Wintersemester 24/25 / \gray{Winter term 24/25}}
\date{}
\maketitle
%
%\begin{center}
\subsubsection*{\deen{Grundlagen solarer Strahlung}{Fundamentals of Solar Radiation}}
%\end{center}
%
\vspace{0.15cm}
Winkelbeziehungen / \gray{Angular relationships}
\begin{fleqn}
\begin{equation}
\begin{split}
\text{\-\ Deklination / \gray{Declination}} \myquad[5.5] \delta = 0.39795 \cdot \cos(0.98563 \cdot (N - 173))
\end{split}
\end{equation}
\end{fleqn}
\begin{fleqn}
\begin{equation}
\begin{split}
\text{\-\ Höhenwinkel / \gray{Altitude angle}} \myquad[2] \sin \gamma = \cos \delta \cdot \cos \Omega_s \cdot \cos \varphi + \sin \delta \cdot \sin \varphi
\end{split}
\end{equation}
\end{fleqn}
\-\ Azimuthwinkel / \gray{Azimuth angle} \\
\begin{equation}
    \sin \alpha_s = 
\begin{cases}
    180^\circ - \frac{\cos \delta \cdot \sin \Omega_s}{\cos \gamma},& \text{wenn / \gray{if} } \cos \Omega_s < \frac{\tan \delta}{\tan \varphi} \\
    \frac{\cos \delta \cdot \sin \Omega_s}{\cos \gamma},& \text{sonst / \gray{otherwise}}
\end{cases}
\end{equation}
\begin{tabular}{ll}
mit / \gray{with} & \\
& \\
Stundenwinkel / \gray{Hour angle} & $\Omega_s$ \\
Breitengrad / \gray{Latitude} & $\varphi$ \\
Fortlaufende Nummerierung der Tage / \gray{Day of year} & $N$ \\
\end{tabular}
\largelinebreak
%
Plancksches Strahlungsgesetz / \gray{Planck's radiation law}
\\
%\begin{equation}
%    B(\lambda, T) = \frac{2hc^2}{\lambda^5} \frac{1}{e^{\frac{hc}{\lambda k_B T}} - 1} 
%\end{equation}
%
\begin{equation}
    E_{\lambda, S}(\lambda, T) = \frac{2\pi hc^2}{\lambda^5 (e^{\frac{hc}{\lambda k_B T}} - 1)}
\end{equation}
\\
%
Wiensches Verschiebungsgesetz / \gray{Wien's displacement law}
\\
\begin{equation}
    \lambda_{\text{max}} T = \SI{2897.8}{\micro\meter\kelvin}
\end{equation}
\\
%
\colorbox{yellow}{Stefan-Boltzmann-Gesetz / \gray{Stefan-Boltzmann law}}
\\
\begin{equation}
    E = \sigma AT^4 
\end{equation}
\\
% \colorbox{yellow}{Intensität und Fluss / \gray{Intensity and flux}}
% \begin{equation}
%     I \propto \frac{1}{r^2}
% \end{equation}
% \begin{equation}
%     I(\theta) = I_0 \cos(\theta)
% \end{equation}
% \colorbox{yellow}{Absorption und Extinktion / \gray{Absorption and extinction}}
% \begin{equation}
%     I = I_0 e^{-\alpha x}
% \end{equation}
Transmissionsgesetz / \gray{Transmission law}
\begin{equation}
    I = I_0 \cdot e^{-\epsilon \cdot s} = I_0 \cdot \tau_G \myquad[3] \text{mit / \gray{with}} \quad \tau_G = \tau_{RS} \cdot \tau_{MS} \cdot \tau_{A}
\end{equation}
\\
Snellius Gesetz / \gray{Snellius' law}
\begin{equation}
    \frac{\sin \alpha}{\sin \beta} = \frac{n_2}{n_1} = \frac{c_1}{c_2}
\end{equation}
Strahlungsintensität / \gray{Radiant intensity}
%\begin{equation}
%    L = \lim_{\Delta \omega \to 0 \newline \Delta A \to 0}\frac{\Delta E}{\Delta A \cdot \Delta \omega}
%\end{equation}
\begin{equation}
    L = \frac{d^2 E}{dA_{\text{proj}} \, d\Omega} = \frac{d^2 E}{\cos \theta \, dA \, d\Omega}
\end{equation} \\
\begin{fleqn}
\begin{equation*}
\begin{split}
\text{\indent mit / \gray{with}} \myquad[5] d\Omega = \sin \theta \, d\theta \, d\varphi \quad \text{(Raumwinkel / \gray{Solid angle})}
\end{split}
\end{equation*}
\end{fleqn}
\\
Strahlungsfluss / \gray{Radiant flux}
%\begin{equation}
%    \Phi = \int L \cos(\theta) d\Omega = \int_{0}^{2\pi} \int_{0}^{\pi} I \cos\Theta \sin\Theta d\Theta d\phi
%\end{equation}
\begin{equation}
    q = \int_{0}^{2\pi} \int_{0}^{\pi} L \cos\theta \sin\theta \, d\theta \, d\varphi
\end{equation}
Etendue / \gray{Etendue}
\begin{equation}
    U := \iint N^2 \cos \theta \, dA \, d\Omega
\end{equation}
%
\newline
\begin{center}
\subsubsection*{Konzentration solarer Strahlung / \gray{Concentration of Solar Radiation}}
\end{center}
%
\vspace{0.15cm}
Brennfleck Paraboloid / \gray{Focal Spot of Paraboloid}
\begin{equation}
    r_{1,B} = r_{t,\alpha D} = \frac{d_{B}}{2} \cdot \alpha_{D}
\end{equation}
\begin{equation}
    r_{2,B} = \frac{r_t \cdot \alpha_D}{\cos(\phi)}
\end{equation}
\begin{equation}
    A = \pi r_t^2 \sin(\phi)^2
\end{equation}
\begin{equation}
    A' = \pi r_t^2 \frac{\sin(\Theta)^2}{\cos(\phi)^2}
\end{equation}
% \begin{equation}
%     C_{\text{max}} = 11550
% \end{equation}
Brennrechteck Parabolrinne / \gray{Focal rectangle of parabolic trough}
\begin{equation}
    r_{B} = \frac{r_t \cdot \alpha_D}{\cos(\phi)} - \frac{d_{B}}{2}
\end{equation}
\begin{equation}
    A = 2 \cdot l \cdot r_t \sin(\phi)
\end{equation}
\begin{equation}
    A' = l \cdot r_t \frac{\sin(\Theta)}{\cos(\phi)}
\end{equation}
% \begin{equation}
%     C_{\text{max}} = 107.5
% \end{equation}
Maximale Temperatur / \gray{Maximum temperature}
%\begin{equation}
%    T_A^4 = T_S^4 \cdot R_S^4 \cdot A_R \cdot C \cdot \sin(\Theta)^2 = T_S^4 \cdot C_{S-E} \cdot A_R
%\end{equation}
\begin{equation}
    T_\text{abs, max} = \sqrt[4]{\frac{I_0}{\sigma}C} = T_S \sqrt[4]{\frac{C}{C_\textbf{max}}}
\end{equation}
% \colorbox{yellow}{Fokusabweichung / \gray{Focus deviation}}
% \begin{equation}
%     FD = 2 \cdot \sigma_{\text{mirror}} \cdot f_{\text{mittl. Fokallaenge}}
% \end{equation}
%
\\
%\begin{center}
\subsubsection*{Thermische Kollektoren / \gray{Thermal Collectors}}
%\end{center}
%
\vspace{0.15cm}
% \colorbox{green}{Grundlagen des Wärmetransfers / \gray{Fundamentals of heat transfer}}\\
% %
% \begin{fleqn}
% \begin{equation}
% \begin{split}
% \text{\-\ Konvektion / \gray{Convection}} \myquad[7] \dot{q}_{\text{conv}} = \alpha \cdot \Delta T = \alpha \cdot (T_{S,2} - T_{\infty})
% \end{split}
% \end{equation}
% \end{fleqn}
% %
% \begin{fleqn}
% \begin{equation}
% \begin{split}
% \text{\-\ Wärmeleitung / \gray{Conduction}} \myquad[5.5] \dot{q}_{\text{cond}} = -\lambda \frac{\Delta T}{L} = -\lambda \frac{T_{S,2} - T_{S,1}}{L}
% \end{split}
% \end{equation}
% \end{fleqn}
% %
% \begin{fleqn}
% \begin{equation}
% \begin{split}
% \text{\-\ Wärmestrahlung / \gray{Radiation}} \myquad[5.5] \dot{q}_{\text{rad}} = \epsilon \cdot \sigma \cdot (T_{S,2}^4 - T_U^4)
% \end{split}
% \end{equation}
% \end{fleqn}
% %
% \linebreak
%
Parabolrinnenkollektoren / \gray{Parabolic Trough Collectors}
%
\linebreak
% \-\ \colorbox{green}{Thermischer Wirkungsgrad / \gray{Thermal efficiency}}
% \begin{equation}
%     \eta_{\text{th}} = \frac{\dot{Q}_{\text{abs}}}{\dot{Q}_{\text{solar}}}
% \end{equation}
% \begin{equation}
%     \dot{Q}_{\text{abs}} = \dot{m} \cdot (h_{\text{out}} - h_{\text{in}})
% \end{equation}
% \begin{equation}
%     \dot{Q}_{\text{solar}} = DNI \cdot A \cdot \cos(\theta)
% \end{equation}
%
\-\ Optischer Wirkungsgrad / \gray{Optical efficiency}
\begin{equation}
    \eta_{\text{opt}} = \rho_{\text{refl}} \cdot \tau_{\text{rec}} \cdot \alpha_{\text{rec}} \cdot IC
\end{equation}
%
\-\ Wirkungsgradgleichung / \gray{Efficiency equation}
\begin{equation}
    \eta_{\text{coll}} = \eta_{\text{opt}} \cdot \kappa(\theta) - c_1 \cdot T_m^* - c_2 \cdot G_b \cdot T_m^{*2}
\end{equation}
%
\begin{fleqn}
\begin{equation*}
\begin{split}
\text{\indent mit / \gray{with}} \myquad[5.5] T_m^* = \frac{T_{\text{fluid}} - T_U}{G_b} \quad \text{and} \quad G_b = DNI \cdot \cos\theta
\end{split}
\end{equation*}
\end{fleqn}
%
% Uncomment if needed
%\textbf{Wärmeverluste im Absorber / Heat Losses in the Absorber}
%\begin{enumerate}
%      \item \text{Strahlungsverluste / Radiation Losses}
%      \begin{equation}
%        \dot{Q}_{\text{rad}} = \sigma \cdot \epsilon_{\text{Gl}} \cdot (T_{\text{Gl}}^4 - T_{\text{U}}^4) \cdot A_{\text{Gl}}
%      \end{equation}
%      \item \text{Konvektive Verluste / Convective Losses}
%      \begin{equation}
%        \dot{Q}_{\text{conv}} = \alpha_{\text{conv}} \cdot (T_{\text{Gl}} - T_{\text{U}}) \cdot A_{\text{Gl}}
%      \end{equation}
%      \item \text{Leitungsverluste / Conduction Losses}
%      \begin{equation}
%        \dot{Q}_{\text{cond}} = \lambda_{\text{cond}} \cdot (T_{\text{Abs}} - T_{\text{Gl}}) \cdot A_{\text{Abs}}
%      \end{equation}
%\end{enumerate}
\-\ Einfallswinkel-Modifikator  / \gray{Incidence Angle Modifier (IAM)}
\begin{equation}
    \kappa(\theta) = b_0 + b_1 \cdot \theta + b_2 \cdot \theta^2 
\end{equation}
\-\ Kollektorausrichtung / \gray{Collector orientation}
%
\begin{fleqn}
\begin{equation}
\begin{split}
\text{\indent N-S Ausrichtung / \gray{N-S orientation}} \myquad[5.5] \cos \theta = \sqrt{1 - \cos^2\gamma \cdot \cos^2\alpha_s} 
\end{split}
\end{equation}
\end{fleqn}
%
\begin{fleqn}
\begin{equation}
\begin{split}
\text{\indent O-W Ausrichtung / \gray{E-W orientation}} \myquad[4.5] \cos\theta = \sqrt{1 - \cos^2\gamma \cdot \sin^2\alpha_s} 
\end{split}
\end{equation}
\end{fleqn}
%
\linebreak
Turmkraftwerk / \gray{Solar tower power plant}
\linebreak
%
\-\ Realer Wirkungsgrad / \gray{Real efficiency}
\begin{equation}
    \eta_{\text{th}} = \frac{\dot{Q}}{A_{\text{Ap}} \cdot C \cdot DNI \cdot \eta_{\text{hel}}}
\end{equation}
%\-\ Heliostatenwirkungsgrad / \gray{Heliostat efficiency}
%\begin{equation}
%    \eta_{\text{hel}} = \frac{E'_{\text{real}}}{E'} = \frac{F'_{\text{real}}}{C \cdot E}
%\end{equation}
\-\ Receiver Bilanz / \gray{Receiver balance}
\begin{equation}
    \dot{Q} = A_{\text{ap}} \left[ \alpha \cdot C \cdot DNI \cdot \eta_{\text{hel}} - \epsilon \cdot \sigma T_{\text{abs}}^4 - U_L \cdot (T_{\text{abs}} - T_{\text{amb}}) \right] = A_{\text{abs}} \cdot U_I \cdot (T_{\text{abs}} - T_{\text{fluid}})
\end{equation}
\newline
%
%\begin{center}
\subsubsection*{Kraftwerksschaltungen / \gray{Power Plant Circuits}}
%\end{center}
%
\vspace{0.15cm}
Thermische Kreisprozesse / \gray{Thermal cycles} \\
\\
% \-\ \colorbox{green}{Carnot-Wirkungsgrad / \gray{Carnot efficiency}}
% \begin{equation}
%     \eta_{\text{Carnot}} = 1 - \frac{T_C}{T_H}
% \end{equation}
\-\ Ideal thermischer Wirkungsgrad Brayton / \gray{Ideal Brayton cycle efficiency}
%\begin{equation}
%    \eta = \frac{h_2 - h_1 + h_4 - h_3}{h_2 - h_1} = 1 - \frac{T_1}{T_2} = 1 - \left(\frac{p_2}{p_1}\right)^{\frac{1-\kappa}{\kappa}}
%\end{equation}
\begin{equation}
    \eta_{th} = 1 - \frac{T_1}{T_2} = 1 - \left(\frac{p_2}{p_1}\right)^{\frac{\kappa - 1}{\kappa}} \quad \text{mit / \gray{with}} \quad \kappa = \frac{c_p}{c_v}
\end{equation}
\-\ Isentrope Zustandsänderung / \gray{Isentropic process}
%\begin{equation}
%    \frac{T_{2,s}}{T_1} = \left(\frac{p_2}{p_1}\right)^{\frac{\kappa - 1}{\kappa}} \quad \text{mit / \gray{with}} \quad \kappa = \frac{c_p}{c_v}
%\end{equation}
\begin{equation}
    \eta_{s,V} = \frac{h_{2,s} - h_1}{h_2 - h_1} \quad \text{und / \gray{and}} \quad \eta_{s,T} = \frac{h_3 - h_4}{h_{3} - h_{4,s}}
\end{equation}
\linebreak
%
\begin{center}
\subsubsection*{Photovoltaik / \gray{Photovoltaics}}
\end{center}
%
\vspace{0.15cm}
PV Zellen / \gray{PV cells} \\
\\
\-\ \colorbox{yellow}{Planck-Einstein Relation / \gray{Planck-Einstein relation}}
\begin{equation}
    E_{\lambda} = \frac{h c}{\lambda}
\end{equation}
%with 
%\begin{equation}
%    c = 2.998 \times 10^8 \, \text{m/s}
%\end{equation}
%\begin{equation}
%    h = 6.626 \times 10^{-34} \, \text{Js}
%\end{equation}
%\begin{equation}
%    1 \, \text{eV} = 1.602 \times 10^{-19} \, \text{J}
%\end{equation}
%\[
%\begin{array}{/c/c/}
%\hline
%\text{Material} & E_G \text{ in eV} \\
%\hline
%\text{Ge} & 0.7 \\
%\text{CuInSe}_2 & 1 \\
%\text{c-Si} & 1.13 \\
%\text{GaAs} & 1.42 \\
%\text{CdTe} & 1.45 \\
%\text{AlSb} & 1.55 \\
%\text{CuGaSe}_2 & 1.7 \\
%\text{a-Si (amorph)} & 1.7 \\
%\text{Al}_{0.85}\text{Ga}_{0.15}\text{As} & 1.9 \\
%\text{GaP} & 2.3 \\
%\text{CdS} & 2.45 \\
%\hline
%\end{array}
%\]
\-\ Shockley-Gleichung / \gray{Shockley equation}
\begin{equation}
    I_D = I_S \left( e^{\frac{U_D}{n U_T}} - 1 \right)
\end{equation}
% \-\ \colorbox{yellow}{Temperaturspannung / \gray{Thermal voltage}}
% \begin{equation}
%     U_T = \frac{k T}{q}
% \end{equation}
\-\ Strom-Spannungskennlinie / \gray{Current-voltage characteristic}
\begin{equation}
    I = I_{\text{ph}} - I_D = c_0 E - I_S \cdot \left(e^{\frac{U_D}{n U_T}} - 1 \right)
\end{equation}
%\begin{equation}
%    P = U \cdot I
%\end{equation}
%\begin{equation}
%    P_{\text{MPP}} = U_{\text{MPP}} \cdot I_{\text{MPP}}
%\end{equation}
\begin{equation}
    FF = \frac{P_{\text{MPP}}}{U_L \cdot I_K} < 1
\end{equation}
\-\ Wirkungsgrad / \gray{Efficiency}
\begin{equation}
    \eta = \frac{P_{\text{el}}}{P_{\text{solar}}} = \frac{P_{\text{MPP}}}{E_{\text{solar}} \cdot A_{\text{cell}}} 
    %= \frac{FF \cdot U_L \cdot I_K}{E_{\text{solar}} \cdot A_{\text{cell}}}
\end{equation}
\-\ Konzentrationsabhängigkeit / \gray{Concentration dependence}
\begin{equation}
    U_L^C = U_L^1 + m \, U_T \, \ln(C)
\end{equation}
\begin{equation}
    I_K^C = C \cdot I_K^1
\end{equation}
%
\newline
%\begin{center}
\subsubsection*{Energiespeicher / \gray{Energy Storage}}
%\end{center}
%
\vspace{0.15cm}
\-\ Speicherkapazität / \gray{Storage capacity}
\begin{equation}
    Q = f_{\text{usage}} \cdot m \cdot \overline{c_p} \cdot (T_{\text{max}} - T_{\text{min}})
\end{equation}
%
\begin{fleqn}
\begin{equation*}
\begin{split}
\text{\indent mit / \gray{with}} \myquad[9] f_{\text{usage}} = \frac{Q_{\text{storage,real}}}{Q_{\text{storage,ideal}}}
\end{split}
\end{equation*}
\end{fleqn}
%
\-\ Katalytische Reaktionen / \gray{Catalytic reactions}
\begin{equation}
    A + B + \Delta^R H^* \rightleftharpoons C + D
\end{equation}
\-\ Thermische Dissoziationsreaktionen / \gray{Thermal dissociation reactions}
\begin{equation}
    \text{AB}_{(s/l)} + \Delta^R H^* \rightleftharpoons \text{A}_{(s/l)} + \text{B}_{(g)}
\end{equation}
\newline
%
\begin{center}
\subsubsection*{Solare Wasserstofferzeugung - \gray{Solar hydrogen production}}
\end{center}
%
\vspace{0.15cm}
\-\ Allgemeine Reaktionsgleichung / \gray{General reaction equation}
\begin{equation}
    \text{C}_n\text{H}_m\text{O}_x + o\text{H}_2\text{O} \rightleftharpoons (n - (o + x))\text{C} + (o + x)\text{CO} + \left(\frac{m}{2} + o\right)\text{H}_2
\end{equation}
\-\ Ideale Gasgleichung / \gray{Ideal gas law}
\begin{equation}
    pV = nRT \quad 
\end{equation}
%
\newpage
%
\begin{center}
\subsubsection*{Konstanten / \gray{Constants}}
\end{center}
%
\vspace{0.15cm}
\-\ Lichtgeschwindigkeit im Vakuum / \gray{Speed of light in vacuum}
    \begin{equation}
        c_0 = \SI{2.998e8}{\metre\per\second}
    \end{equation}
%
\-\ Planck-Konstante / \gray{Planck's constant}
    \begin{equation}
        h = \SI{6.626e-34}{\joule\second}
    \end{equation}
%
\-\ Elementarladung eines Elektrons / \gray{Electronvolt}
    \begin{equation}
        \SI{1}{\electronvolt} = \SI{1.602e-19}{\joule}
    \end{equation}
%
\-\ Ideale Gaskonstante / \gray{Ideal gas constant}
    \begin{equation}
        R = \SI{8.314}{\joule\per\mole\per\kelvin}
    \end{equation}
%
\-\ Boltzmann-Konstante / \gray{Boltzmann's constant}
    \begin{equation}
        k_B = \SI{1.381e-23}{\joule\per\kelvin}
    \end{equation}
%
\-\ {Stefan-Boltzmann-Konstante / \gray{Stefan-Boltzmann constant}
    \begin{equation}
        \sigma = \SI{5.67e-8}{\watt\per\metre\squared\per\kelvin\tothe{4}}
    \end{equation}
%
\linebreak
%\begin{center}
\subsubsection*{Bandlücken / \gray{Band Gaps}}
%\end{center}
%
\vspace{0.15cm}


\begin{table}[h!]
\begin{center}
\caption{Bandlücken}\label{tab:meine}
\begin{tabular}{lc} \toprule
{Material} & {$E_G$ \textbf{[\si{\electronvolt}]}} \\ \midrule
\ch{Ge} & 0.70 \\
\ch{CuInSe2} & 1.00 \\
\ch{c-Si} & 1.13 \\
\ch{GaAs} & 1.42 \\
\ch{CdTe} & 1.45 \\
\ch{AlSb} & 1.55 \\
\ch{CuGaSe2} & 1.70 \\
\ch{a-Si_{(amorph)}} & 1.70 \\
\ch{Al0.85Ga0.15As} & 1.90 \\
\ch{GaP} & 2.30 \\
\ch{CdS} & 2.45 \\ \bottomrule
\end{tabular}
\end{center}
\end{table}



%\begin{table}[h!]
%    \centering
%    \begin{tabular}{/c/c/}
%        \hline
%        \textbf{Material} & $E_G$ \textbf{[\si{\electronvolt}]} \\
%        \hline
%        \ch{Ge} & 0.70 \\
%        \ch{CuInSe2} & 1.00 \\
%        \ch{c-Si} & 1.13 \\
%        \ch{GaAs} & 1.42 \\
%        \ch{CdTe} & 1.45 \\
%        \ch{AlSb} & 1.55 \\
%        \ch{CuGaSe2} & 1.70 \\
%        \ch{a-Si_{(amorph)}} & 1.70 \\
%        \ch{Al0.85Ga0.15As} & 1.90 \\
%        \ch{GaP} & 2.30 \\
%        \ch{CdS} & 2.45 \\
%        \hline
%    \end{tabular}
%\end{table}



\end{document}
