\documentclass[12pt,ngerman]{beamer}
\usepackage[utf8]{inputenc}
\usepackage[T1]{fontenc}
\usepackage{booktabs}
\usepackage{babel}
\usepackage{graphicx}
\usepackage{csquotes}
\usepackage{xcolor}
 
 
\usepackage[sfdefault]{plex-sans}
\usetheme[progressbar=frametitle]{metropolis}           % Use metropolis theme
 
\title{A minimal Metropolis + IBM Plex example}
\date{\today}
\author{Max Mustermann}
\institute{The Name of the Institute}
 
\makeatletter
\setlength{\metropolis@titleseparator@linewidth}{1pt}
\setlength{\metropolis@progressonsectionpage@linewidth}{1pt}
\setlength{\metropolis@progressinheadfoot@linewidth}{1pt}
\makeatother
 
\begin{document}
 
\begin{frame}
	 \maketitle
\end{frame}
 
\begin{frame}
\frametitle{Introduction 1}
\framesubtitle{~}
 
\[ a^2 + b^2 = c^2 \]
 
\begin{itemize}
\item 
\item 
\item 
\item 
\item 
\item 
\end{itemize}
\end{frame}
 
\begin{frame}
\frametitle{Introduction 2}
\framesubtitle{~}
 
\[ a^2 + b^2 = c^2 \]
 
\begin{itemize}
\item 
\item 
\item 
\item 
\item 
\item 
\end{itemize}
\end{frame}
 
 \begin{frame}
\frametitle{Introduction 3}
\framesubtitle{~}
 
\[ a^2 + b^2 = c^2 \]
 
\begin{itemize}
\item 
\item 
\item 
\item 
\item 
\item 
\end{itemize}
\end{frame}
 
 
 \begin{frame}
\frametitle{Introduction 4}
\framesubtitle{~}
 
\[ a^2 + b^2 = c^2 \]
 
\begin{itemize}
\item 
\item 
\item 
\item 
\item 
\item 
\end{itemize}
\end{frame}
 
 
 \begin{frame}
\frametitle{Introduction 5}
\framesubtitle{~}
 
\[ a^2 + b^2 = c^2 \]
 
\begin{itemize}
\item 
\item 
\item 
\item 
\item 
\item 
\end{itemize}
\end{frame}
 
 
 \begin{frame}
\frametitle{Introduction 6}
\framesubtitle{~}
 
\[ a^2 + b^2 = c^2 \]
 
\begin{itemize}
\item 
\item 
\item 
\item 
\item 
\item 
\end{itemize}
\end{frame}
 
 
 \begin{frame}
\frametitle{Introduction 7}
\framesubtitle{~}
 
\[ a^2 + b^2 = c^2 \]
 
\begin{itemize}
\item 
\item 
\item 
\item 
\item 
\item 
\end{itemize}
\end{frame}
  
 
 
\end{document}