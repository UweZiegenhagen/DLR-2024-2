% https://tex.stackexchange.com/questions/59742/progress-bar-for-latex-beamer
\documentclass[12pt,ngerman]{beamer}
\usepackage[utf8]{inputenc}
\usepackage[T1]{fontenc}

 
 
\title{TSX Progressbar}
\date{\today}
\author{Max Mustermann}
\institute{The Name of the Institute}
 

\usepackage{tikz}
\usetikzlibrary{calc}

\definecolor{pbblue}{HTML}{0A75A8}% color for the progress bar and the circle

\makeatletter
\def\progressbar@progressbar{} % the progress bar
\newcount\progressbar@tmpcounta% auxiliary counter
\newcount\progressbar@tmpcountb% auxiliary counter
\newdimen\progressbar@pbht %progressbar height
\newdimen\progressbar@pbwd %progressbar width
\newdimen\progressbar@rcircle % radius for the circle
\newdimen\progressbar@tmpdim % auxiliary dimension

\progressbar@pbwd=\linewidth
\progressbar@pbht=1pt
\progressbar@rcircle=2.5pt

% the progress bar
\def\progressbar@progressbar{%

    \progressbar@tmpcounta=\insertframenumber
    \progressbar@tmpcountb=\inserttotalframenumber
    \progressbar@tmpdim=\progressbar@pbwd
    \multiply\progressbar@tmpdim by \progressbar@tmpcounta
    \divide\progressbar@tmpdim by \progressbar@tmpcountb

  \begin{tikzpicture}
    \draw[pbblue!30,line width=\progressbar@pbht]
      (0pt, 0pt) -- ++ (\progressbar@pbwd,0pt);

    \filldraw[pbblue!30] %
      (\the\dimexpr\progressbar@tmpdim-\progressbar@rcircle\relax, .5\progressbar@pbht) circle (\progressbar@rcircle);

    \node[draw=pbblue!30,text width=3.5em,align=center,inner sep=1pt,
      text=pbblue!70,anchor=east] at (0,0) {\insertframenumber/\inserttotalframenumber};
  \end{tikzpicture}%
}

\addtobeamertemplate{headline}{}
{%
  \begin{beamercolorbox}[wd=\paperwidth,ht=4ex,center,dp=1ex]{white}%
    \progressbar@progressbar%
  \end{beamercolorbox}%
}
\makeatother

 
\begin{document}
 
\begin{frame}
	 \maketitle
\end{frame}
 
\begin{frame}
\frametitle{Introduction 1}
\framesubtitle{~}
 
\[ a^2 + b^2 = c^2 \]
 
\begin{itemize}
\item 
\item 
\item 
\item 
\item 
\item 
\end{itemize}
\end{frame}
 
\begin{frame}
\frametitle{Introduction 2}
\framesubtitle{~}
 
\[ a^2 + b^2 = c^2 \]
 
\begin{itemize}
\item 
\item 
\item 
\item 
\item 
\item 
\end{itemize}
\end{frame}
 
 \begin{frame}
\frametitle{Introduction 3}
\framesubtitle{~}
 
\[ a^2 + b^2 = c^2 \]
 
\begin{itemize}
\item 
\item 
\item 
\item 
\item 
\item 
\end{itemize}
\end{frame}
 
 
 \begin{frame}
\frametitle{Introduction 4}
\framesubtitle{~}
 
\[ a^2 + b^2 = c^2 \]
 
\begin{itemize}
\item 
\item 
\item 
\item 
\item 
\item 
\end{itemize}
\end{frame}
 
 
 \begin{frame}
\frametitle{Introduction 5}
\framesubtitle{~}
 
\[ a^2 + b^2 = c^2 \]
 
\begin{itemize}
\item 
\item 
\item 
\item 
\item 
\item 
\end{itemize}
\end{frame}
 
 
 \begin{frame}
\frametitle{Introduction 6}
\framesubtitle{~}
 
\[ a^2 + b^2 = c^2 \]
 
\begin{itemize}
\item 
\item 
\item 
\item 
\item 
\item 
\end{itemize}
\end{frame}
 
 
 \begin{frame}
\frametitle{Introduction 7}
\framesubtitle{~}
 
\[ a^2 + b^2 = c^2 \]
 
\begin{itemize}
\item 
\item 
\item 
\item 
\item 
\item 
\end{itemize}
\end{frame}
  
 
 
\end{document}