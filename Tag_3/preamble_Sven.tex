\documentclass[a4paper,twoside,ngerman,12pt,headsepline,footsepline,plainfootsepline,numbers=noendperiod]{scrbook}

\usepackage{luatex85,shellesc}	%Kompatibilität für Lualatex als PDF generator und Shell escape Anwendung
\usepackage{lmodern}	%Veränderung der Schriftart auf latin modern (sieht im pdf besser aus)
\usepackage[ngerman]{babel}		%Multilingualsupport
\usepackage{csquotes}
\usepackage{scrlayer-scrpage}	%Package für Style Header
\usepackage{subcaption}		%Subcaption für Bilder
\usepackage[format=hang,singlelinecheck=on,labelsep=colon,labelformat=simple,figurename=Abb.,tablename=Tab.]{caption}	% Formatierung der Unterschrift für Bilder, Tabellen



%%%%%%%%%%%%%%%%%%%%%%%%%%%%%%%%%%%%%%%%%%%%%%%%%%%%%%%%%%%%%%%%%%%%%%%%%%%%%%%%%%
% === Literaturverzeichnis === %%%%%%%%%%%%%%%%%%%%%%%%%%%%%%%%%%%%%%%%%%%%%%%%%%%
%%%%%%%%%%%%%%%%%%%%%%%%%%%%%%%%%%%%%%%%%%%%%%%%%%%%%%%%%%%%%%%%%%%%%%%%%%%%%%%%%%
% Emfpehlung: BIBER NUTZEN statt bibtex nutzen!!!

%\usepackage[style=numeric-comp]{biblatex}
%\usepackage[backend=biber,style=alphabetic]{biblatex} % BibLaTeX Paket einbinden.
\usepackage[
	backend=biber,
	style=numeric,
	sorting=none,
	sortcites=true
]{biblatex}
% Erläuterungen zu den Optionen:
% - backend=biber: Gibt an, dass biber als Backend für die Verarbeitung der Bibliographie verwendet wird (Standard und empfohlen).
% - style=numeric: Verwendet numerische Zitate.
% - sorting=none: Stellt sicher, dass die Zitate in der Reihenfolge erscheinen, in der sie im Text zitiert werden.
% - sortcites=true: Sortiert und fasst aufeinanderfolgende Zitate zusammen.

% === Folgendes geht NICHT mit BibLaTeX, weil das Paket "natbib" damit nicht kompatibel ist ===
%\usepackage[numbers]{natbib}	% Wenn man alphabetische Zitate möchte, kann die numbers-Option weglassen werden.
%\usepackage[numbers,sort&compress]{natbib}	% Mit der "sort&compress"-Option werden die Zitate im Text automatisch in aufsteigender Reihenfolge sortiert und zusammengefasst, wenn sie direkt nacheinander zitiert werden. Zum Beispiel wird \cite{key3,key1,key2} im Text als [1-3] angezeigt, vorausgesetzt, dass key1, key2 und key3 aufeinanderfolgende Nummern sind.

% === Einbinden der .bib Datei ===
\addbibresource{data/lib.bib} 	% \addbibresource muss für BibLaTeX (biber) verwendet werden!
%\bibliography{data/lib}		% \bibliography funktioniert NICHT für BibLaTeX (biber)!

%%%%%%%%%%%%%%%%%%%%%%%%%%%%%%%%%%%%%%%%%%%%%%%%%%%%%%%%%%%%%%%%%%%%%%%%%%%%%%%%%%
% === Ende Literaturverzeichnis === %%%%%%%%%%%%%%%%%%%%%%%%%%%%%%%%%%%%%%%%%%%%%%
%%%%%%%%%%%%%%%%%%%%%%%%%%%%%%%%%%%%%%%%%%%%%%%%%%%%%%%%%%%%%%%%%%%%%%%%%%%%%%%%%%


\usepackage{pdfpages}	%Einbinden von PDF Seiten
\usepackage{mathtools}	%mathematische Symbole
\usepackage{amsmath, amsthm, amssymb}	%Erweiterungen für matehmatische Symbole
\usepackage{tikz}	%Bereitstellung Figuren und Zeichnen
\usetikzlibrary{plotmarks,shapes,arrows.meta,decorations.pathmorphing,patterns,arrows}
\usepackage{pgfplots}	%Bereitstellung PLots
\pgfplotsset{
	/pgf/number format/.cd,
	use comma,
	1000 sep={}
}

% === Tabellen ===
\usepackage{multirow}	%Bearbeitung von Tabellen
\usepackage{pgfplotstable} %für Datentabellen
\usepackage{longtable}		%Für lange Tabellen


%\usepackage[showframe, nomarginpar]{geometry}
\usepackage[titletoc]{appendix}
%Erstellung für Formel und Abkürzungsverzeichnis
%\usepackage[automake,nomain,nonumberlist,acronym,xindy={language=german,codepage=duden-utf8},toc]{glossaries}
\usepackage[style=alttree,nonumberlist,section=subsection,nopostdot,nogroupskip,acronym]{glossaries}
\newglossary{sym_list_latin}{gls1}{glo1}{lateinische Symbole}
\newglossary{sym_list_greek}{gls2}{glo2}{griechische Symbole}
%\setacronymstyle{long-sc-short}
\renewcommand{\glsglossarymark}[1]{}	%Einzelne Glossaries erzeugen keine Marke mehr, die im Header auftaucht. Stattdessen steht überall dann der Teil aus \chapter. In dem Fall dann "Nomenklatur"

%Paket für Zeilenabstand
\usepackage{setspace}
\onehalfspacing

%------------Externalize
\pgfplotsset{width=7cm,compat=newest,plot coordinates/math parser=false}
\usepgfplotslibrary{external}
\tikzexternalize[prefix=TikzPics/,shell escape=-enable-write18]%, optimize=false] %/Ordner ist TikzPics
%------------Externalize

%------------Header
\automark[chapter]{chapter}
\clearpairofpagestyles	%Ist das Neue für das veraltete: \clearscrheadfoot.
\ofoot[\pagemark]{\pagemark}
\ohead{\headmark}
\pagestyle{scrheadings}
%------------Header

%Einstellung für Text 2,5cm oben, 2cm unten, 3 cm links, 3 cm rechts, Kopf- und Fußzeile jeweils 1,25cm vom Seitenrand
\hoffset 		= -10mm			%Abstand links 1inch + 0.46cm = 30mm
\voffset 		= -10.5mm		%Abstand oben 1inch-12.9mm = 12.5mm; Messung hat ergeben, dass der Header bei 10 mm ist statt bei 12.5mm
\oddsidemargin 	= 20mm			%abstand vom rand (1inch+diesen wert); wird nicht benötigt da durch hoffset geklärt
\topmargin 		= 0mm			%Abstand der 25 mm zum header
\headheight 	= 6.3mm			%Höhe header
\headsep 		= 5.7mm			%abstand header zu body
\textheight 	= 245mm			%Höhe des textbodys
\textwidth 		= 150mm			%Breite des Textbodys
\marginparsep 	= 0cm			%Abstand von Textbodybreite zum Randnotizfeld
\marginparwidth = 0cm			%Breite des Randnotizfeldes
\footskip 		= 13mm			%Abstand von Textbodyhöhe zur Fußnote

