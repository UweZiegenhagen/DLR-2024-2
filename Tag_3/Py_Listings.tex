\documentclass[a4paper,ngerman,12pt]{scrartcl} 
 
\usepackage{babel}
\usepackage[T1]{fontenc}
\usepackage[math]{iwona}
\usepackage{xcolor}

\definecolor{hellgelb}{rgb}{1,1,0.8}
\definecolor{lightgelb}{rgb}{1,1,0.8}
\definecolor{colKeys}{rgb}{0,0,1}
\definecolor{colIdentifier}{rgb}{0,0,0}
\definecolor{colComments}{rgb}{1,0,0}
\definecolor{colString}{rgb}{0,0.5,0}

\usepackage{listings} % <===== 
 
\lstset{% Mein Standard Listings template
    float=hbp,%
    basicstyle=\ttfamily\footnotesize, %
    identifierstyle=\color{colIdentifier}, %
    keywordstyle=\color{colKeys}, %
    stringstyle=\color{colString}, %
    commentstyle=\color{colComments}, %
    columns=flexible, %
    tabsize=2, %
    frame=single, %
    upquote=true,%
    extendedchars=true, %
    showspaces=false, %
    showstringspaces=false, %
    numbers=left, %
    numberstyle=\tiny, %
    breaklines=true, %
    language={Python},
    morekeywords={choice,remove},
    backgroundcolor=\color{hellgelb}, %
    breakautoindent=true, %
    captionpos=b%
}
 
 
%%%%%%%%%%%% für Umlaute in Listings
\lstset{literate=%
    {Ö}{{\"O}}1
    {Ä}{{\"A}}1
    {Ü}{{\"U}}1
    {ß}{{\ss}}1
    {ü}{{\"u}}1
    {ä}{{\"a}}1
    {ö}{{\"o}}1
    {~}{{\textasciitilde}}1
}
 

\begin{document}
 
 
\begin{lstlisting}
def create_bruch():
    zahlen = list(range(1,13))
    zaehler = random.choice(zahlen)
    zahlen.remove(zaehler)
    nenner = random.choice(zahlen)
    return 1234
\end{lstlisting}
 
\begin{lstlisting}
def create_bruch():
    zahlen = list(range(1,13))
    zaehler = random.choice(zahlen)
    zahlen.remove(zaehler)
    nenner = random.choice(zahlen)
    return 1234
\end{lstlisting}

\clearpage

\lstinputlisting[caption={Mein Listing},linerange={1-6,20-22}]{paramikooo.py}


\end{document}





 

 

 
 