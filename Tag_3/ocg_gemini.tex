% Beispiel funktioniert nicht
\documentclass{article}
\usepackage[german]{babel}

\usepackage{ocgx}
\usepackage{tikz}
\usepackage{xstring}

\parindent=0pt

\title{Beispieldokument  
 mit OCG}
\author{Dein Name}
\date{\today}

\usepackage{hyperref}
\begin{document}
\maketitle

% Definiere einen Counter, der jeweils die maximale Anzahl Zeilen in einem gelayerten Block zählt
\newcounter{maxzeilen}

% Definiere die verschiedenen Layer, ihre interen  
 Abkürzungen und ob sie am Anfang sichtbar sind ({1} oder {0})
\begin{ocg}{Layer 1}{layer1}{1}
  Dieser Text ist im Layer 1.
\end{ocg}

\begin{ocg}{Layer 2}{layer2}{0}
  Dieser Text ist im Layer 2 und ist anfangs ausgeblendet.
\end{ocg}

% Ein Beispiel mit einem TikZ-Bild, das in einem Layer enthalten ist
\begin{ocg}{TikZ-Bild}{tikzbild}{0}
\begin{tikzpicture}
  \draw (0,0) -- (2,2);
\end{tikzpicture}
\end{ocg}

% Ein Beispiel mit einem Link, der einen Layer ein- oder ausblendet
\hyperref[layer2]{Hier klicken, um Layer 2 ein- oder auszublenden}

\end{document}